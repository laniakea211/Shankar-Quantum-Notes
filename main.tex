\documentclass[10pt,twocolumn,leqno]{article}
\usepackage{amsmath,amsfonts,amssymb, mathtools}
\usepackage{physics, sectsty, framed, hyperref, bm}
\usepackage{amsthm}     % Std., but Prof. King uses a custom pkg, instead.
\usepackage{enumerate}  % Easy generation of fancy enumerated-lists.
\usepackage{lastpage}
\usepackage{relsize}    % Provides cmds `\smaller' and `\larger'.
\usepackage{tikz}       % For drawing

%% T.fol is commented-out.  Use if you want to rotate text/images, or do
%% fancy stuff with boxed text/equations .

% \usepackage{graphicx}
%% T.fol gives many macros for coloring text.
\usepackage{xcolor}

%% Alternatively, you could use this older package:
% \usepackage{color}
\usepackage{url}
\def\eddress{\begingroup \urlstyle{tt}\Url}

%!! Was called "~/Classwork/Analysis/kelly-f-hm.tex".

                        
\author{{Erin Strickland} \\
    \eddress{strickland.g.erin@gmail.com}}
\title{{\vspace*{-1.5in}}Notes on Shankar's \\
    \emph{Principles of Quantum Mechanics}}
\date{\vspace{-2ex}}
%%

                        %%% STA: Abbreviations %%%
%%%% Tools
\newcommand{\nc}{\newcommand}   \nc{\renc}{\renewcommand} \newcommand{\pc}{\providecommand}
%%
\nc\jq[1]{``\nobreak#1\unskip''} \nc{\jQ}[1]{`\nobreak#1\unskip'} % Quote
\nc\df[1]{\textit{\textbf{#1}}} %DeFining a new term.
\nc\emff{\textbf} % Strong emphasis. (Forte)
\nc\emf{\textit} % Emphasis.  Does not switch between upshape and
                        % itshape, unlike \emph.
\nc\emfu[1]{\underbar{\ignorespaces#1\unskip}}  %Underlined emphasis.
\nc\emfc[1]{{\color{red}#1}} %Emphasis by color.
\nc{\emfff}[1]{\emfu{\emff{#1}}}
\nc\thm[1]{{\mdseries\textsf{{#1}}}}    %For *citing* a thm by name.
\nc\nth[2][th]{\ensuremath{#2^{\!\text{\textit{\,#1}}}}}
%%
% Note: Don't confuse \naturals with \natural --which is the prefix for a
% "natural" in musical notation.
\nc\zed{\mathbb{Z}}     \nc\zplus{{\zed_+}}     \nc\zminus{{\zed_-}}
\nc\naturals{\mathbb{N}}
%
\nc\Q{{\mathbb{Q}}}     \nc\qplus{{\Q_+}}       \nc\qminus{{\Q_-}}
\nc\QDyadic{{\mathbb{D}}} %Dyadic rationals.
%
\nc\reals{\mathbb{R}}    \nc\rplus{{\real_+}}    \nc\rminus{{\real_-}}
\nc\realonept{{\dot\real}}      %One-point cptification of |R.
\nc\realtwopt{{\overline\real}} %Two-point cptification of |R.
% Alternatively:   \nc\realtwopt{{\ddot\real}}
% T.preceding also called "Extended reals".
\nc\realtwoptset{{\rcc\mii\pii}}
%
\nc\complex{\mathbb{C}} %Complex numbers.
%%%% Kelly's and Erin's definitions
\nc\g{g}        \nc\gx{g(x)}            \nc\fx{f(x)}
\nc\fn{{f_n}}   \nc\fnx{{{f_n}(x)}}     \nc\kron{\delta_{ij}}
%
\nc\vecsp{\mathbb{V}} % Vector space
                        %%% END: Abbreviations %%%
                        
                        %%% STA: Modifiers %%%
%%
% Set colors for section headings
\sectionfont{\color{violet}}
\subsectionfont{\color{violet}}
\subsubsectionfont{\color{violet}}

% Set theorem styles
\newtheoremstyle{theorem}
    {}
    {}
    {\color{blue}}
    {}
    {\color{red}}
    {}
    { }
    {}
\theoremstyle{theorem}
\newtheorem{theorem}{Theorem}
\newtheorem{lemma}{Lemma}
\newtheorem*{claim}{Claim}
\newtheorem{corollary}{Corollary}
\newtheorem{proposition}[theorem]{Proposition}
\newtheorem*{remark}{Remark}

\newtheoremstyle{example}
    {}
    {}
    {}
    {}
    {\color{olive}}
    {}
    { }
    {}
\theoremstyle{example}
\newtheorem{example}{Example}[subsection]

\newtheoremstyle{exercise}
    {}
    {}
    {\color{teal}}
    {}
    {\color{red}}
    {}
    { }
    {}
\theoremstyle{exercise}
\newtheorem{exercise}{Exercise}[subsection]

% Redefine \proof
\let\oldproof\proof
\renc{\proof}{\oldproof[\emfc{Proof.}]}
\renc\qedsymbol{\emfc{$\lozenge$}}

% Have equations numbered by subsection
\numberwithin{equation}{subsection}

% Table of contents depth
\setcounter{tocdepth}{2}
%%
                        %%% END: Modifiers %%%                        

%% Document
\begin{document}

\maketitle

\tableofcontents

\section{Mathematical Introduction}
\subsection{Linear Vector Spaces: Basics}

\colorlet{shadecolor}{pink}
\begin{shaded*}
A \df{linear vector space} $\vecsp$ is a collection of objects $\ket{1}, \ket{2}, \dots, \ket{V}, \dots, \ket{W}, \dots$, called vectors, for which there exists a definite rule for forming the vector sum (denoted $\ket{V} + \ket{W}$) and a definite rule for multiplication by scalars $a, b, \dots$, (denoted $a\ket{V}$) with the following features:
    \begin{enumerate}
        \item Closure: $\ket{V} + \ket{W} \in \vecsp$ and $a\ket{V} \in \vecsp$,
        \item Scalar multiplication is distributive in the vectors: $a(\ket{V} + \ket{W}) = a\ket{V} + a\ket{W}$,
        \item Scalar multiplication is distributive in the scalars: $(a + b)\ket{V} = a\ket{V} + b\ket{V}$,
        \item Scalar multiplication is associative: \\$a(b\ket{V}) = ab\ket{V}$,
        \item Addition is commutative: \\$\ket{V} + \ket{W} = \ket{W} + \ket{V}$,
        \item Addition is associative: \\$\ket{V} + (\ket{W} + \ket{Z}) = (\ket{V} + \ket{W}) + \ket{Z}$,
        \item There exists a null vector $\ket{0}$ obeying \\$\ket{V} + \ket{0} = \ket{V}$,
        \item For every vector $\ket{V}$ there exists an inverse under addition, $\ket{-V}$, such that \\$\ket{V} + \ket{-V} = \ket{0}$.
    \end{enumerate}
\end{shaded*}


The numbers $a, b, \dots$ are called the \df{field} over which the vector space is defined.

\begin{exercise}
Prove the following: (a) $\ket{0}$ is unique, (b) $0\ket{V} = \ket{0}$, (c) $\ket{-V} = -\ket{V}$, (d) $\ket{-V}$ is the unique additive inverse of $\ket{V}$.
\end{exercise}

%\newpage
\begin{enumerate}
    \item[] \begin{proof}
        \item[(a)] Suppose $\ket{0}$ and $\ket{0'}$ are null vectors. Then we have $\ket{0} + \ket{0'} = \ket{0}$ and $\ket{0'} + \ket{0} = \ket{0'}$ by the definition of the null vector. Thus, by commutativity of addition, $\ket{0} = \ket{0'}$.
        \item[(b)] We have
        \begin{align*}
            \ket{0} &= (0 + 1)\ket{V} + \ket{-V} \\
            &= 0\ket{V} + \ket{V} + \ket{-V} \\
            &= 0\ket{V}.
        \end{align*}
        \item[(c)] We have $\ket{V} + (-\ket{V}) = 0\ket{V} = \ket{0}$. Adding $\ket{-V}$ to each side of the equation yields \\$-\ket{V} = \ket{-V}$.
        \item[(d)] Suppose $\ket{W}$ is an additive inverse of $\ket{V}$. Then $\ket{V} + \ket{W} = \ket{0}$. But $\ket{0}$ is unique, and so $\ket{V} + \ket{W} = \ket{V} + \ket{-V}$. Again adding $\ket{-V}$ to each side of this equation yields $\ket{W} = \ket{-V} $.
    \end{proof}
\end{enumerate}

\begin{exercise}
Consider the set of all entries of the form $(a,b,c)$ where the entries are real numbers. Addition and scalar multiplication are defined as follows: $$(a, b, c) + (d, e, f) = (a + d, b + e, c + f),$$ $$\alpha(a, b, c) = (\alpha a, \alpha b, \alpha c).$$ Write down the null and inverse of $(a, b, c)$ and show that vectors of the form $(a, b, 1)$ do not form a vector space.
\end{exercise}

\begin{proof}
    The null vector of $(a, b, c)$ is $(0, 0, 0)$, and its inverse is $(-a, -b, -c)$. Vectors of the form $(a, b, 1)$ do not form a vector space under the given operations because they are not closed under addition (the third component will be 2, which does not have the required form).
\end{proof}

\begin{example}
The set of $2 \times 2$ matrices form a vector space.
\end{example}

\begin{example}
Consider all functions $\fx$ defined in an interval $0 \leq x \leq L$. With scalar multiplication by $a$ defined as $a\fx$ and vector addition defined as $(f + g)(x) = \fx + \gx$, this forms a vector space.
\end{example}

\begin{exercise}
Do the functions that vanish at the endpoints form a vector space? How about periodic functions obeying $f(0) = f(L)$? How about functions that obey $f(0) = 4$?
\end{exercise}

\begin{proof}
The functions which vanish at the endpoints do form a vector space, since adding any two vectors together will result in the endpoints still vanishing. Similarly for scalar multiplication. Periodic functions obeying $f(0) = f(L)$ also form a vector space, since $f(0) + g(0) = f(L) + g(L)$ and $af(0) = af(L)$. Functions that obey $f(0) = 4$ do not form a vector space since $f(0) + g(0) = 8 \neq 4$ and $af(0) = 4a = 4 \iff a = 1$.
\end{proof}

\begin{shaded*}
Consider a linear relation of the form
\begin{equation}\label{linear independence relation}
    \sum_{i = 1}^n a_i \ket{i} = \ket{0},
\end{equation}
where $\ket{i} \neq \ket{0}$. The set of vectors is said to be \df{linearly independent} if the only linear relation such as \eqref{linear independence relation} is the trivial one with all $a_i = 0$. If the set of vectors is not linearly independent, we say they are \df{linearly dependent}.
\end{shaded*}

To be linearly independent means you can't write any member of the set in terms of the others. Otherwise you could:
\begin{equation*}
    \ket{3} = \sum_{i = 1, \neq 3}^n \frac{-a_i}{a_3}\ket{i},
\end{equation*}
supposing $a_3 \neq 0$, for example.

\begin{exercise}
Consider three elements from the vector space of real $2 \times 2$ matrices:
$$
\ket{1} = 
\begin{bmatrix}
0 & 1 \\
0 & 0
\end{bmatrix}
\quad
\ket{2} = 
\begin{bmatrix}
1 & 1 \\
0 & 1
\end{bmatrix}
\quad
\ket{3} = 
\begin{bmatrix}
-2 & -1 \\
0  & -2
\end{bmatrix}.
$$
Are they linearly independent?
\end{exercise}

\begin{proof}
Suppose $\sum_{i = 1}^3 a_i \ket{i} = 0.$ Then we have
\begin{align*}
    a_1(0) + a_2(1) + a_3(-2) &= 0\\
    a_1(1) + a_2(1) + a_3(-1) &= 0\\
    a_1(0) + a_2(0) + a_3(0)  &= 0\\
    a_1(0) + a_2(1) + a_3(-2) &= 0.
\end{align*}
Thus $a_2 = 2a_3$, so $a_1 + a_3 = 0 \implies a_1 = -a_3$. And so these matrices are not linearly independent.
\end{proof}

\begin{exercise}
Show that the following row vectors are linearly dependent: $(1, 1, 0)$, $(1, 0, 1)$, and $(3, 2, 1)$. Show the opposite for $(1, 1, 0)$, $(1, 0, 1)$, and $(0, 1, 1)$.
\end{exercise}

\begin{proof}
The first three are linearly dependent because $(3, 2, 1) = 2(1, 1, 0) + (1, 0, 1)$. To see that the next three are linearly independent, consider the equation $a(1, 1, 0) + b(1, 0, 1) + c(0, 1, 1) = 0$. Then $a = -b$, $b = -c$, and $a = -c$. But then $b = c$, and since $b = -c$, we have that $a = b = c = 0$.
\end{proof}

\begin{shaded*}
A vector space has \df{dimension $n$} if it can accommodate a maximum of $n$ linearly independent vectors. It will be denoted by $\vecsp^n(\reals)$ if the field is real and by $\vecsp^n(\complex)$ if the field is complex.
\end{shaded*}

\begin{example}
The set of $2 \times 2$ matrices form a four-dimensional vector space. The proof is that the following vectors are linearly independent:
\begin{align*}
&\ket{1} = 
\begin{bmatrix}
1 & 0 \\
0 & 0
\end{bmatrix}
\quad
\ket{2} = 
\begin{bmatrix}
0 & 1 \\
0 & 0
\end{bmatrix}
\\
&\ket{3} = 
\begin{bmatrix}
0 & 0 \\
1 & 0
\end{bmatrix}
\quad
\ket{4} = 
\begin{bmatrix}
0 & 0 \\
0 & 1
\end{bmatrix}.
\end{align*}
Moreover, any arbitrary $4 \times 4$ matrix can be written in terms of them, and so the dimensionality of the space can't be greater than 4.
\end{example}

\begin{theorem}\label{linear basis theorem}
Any vector $\ket{V}$ in an $n$-dimensional space can be written as a linear combination of $n$ linearly independent vectors $\ket{1}, \dots, \ket{n}$.
\end{theorem}

\begin{proof}
If there were a vector $\ket{V}$ for which this were not possible, it would join the given set of vectors and form a set of $n + 1$ linearly independent vectors, which is not possible in an $n$-dimensional space by definition.
\end{proof}

\begin{shaded*}
A set of $n$ linearly independent vectors in an $n$-dimensional space is called a \df{basis}. Thus, for any $\ket{V}$, we can write
\begin{equation}\label{basis expansion}
    \boxed{\ket{V} = \sum_{i = 1}^n v_i \ket{i},}
\end{equation}
where the vectors $\ket{i}$ form a basis.
\end{shaded*}

The coefficients of expansion $v_i$ of a vector in terms of a linearly independent basis are called the \df{components of the vector in that basis}.

\begin{theorem}\label{unique basis expansion}
The expansion in \eqref{basis expansion} is unique.
\end{theorem}

\begin{proof}
Suppose 
\begin{equation*}
    \ket{V} = \sum_{i = 1}^n v_i \ket{i}
            = \sum_{i = 1}^n v'_i \ket{i}.
\end{equation*}
Then
\begin{equation*}
    \ket{0} = \sum_{i = 1}^n (v_i - v'_i) \ket{i},
\end{equation*}
which implies that $v_i = v'_i$ since the basis vectors are linearly independent.
\end{proof}

\begin{remark}
If you change the basis, the components will change as well. We refer to the vector $\ket{V}$ in the abstract, but give it a concrete form when we expand it in a particular basis.
\end{remark}

To add two vectors, add their components:
\begin{equation}\label{component vector addition}
    \boxed{\ket{V} + \ket{W} = \sum_i (v_i + w_i) \ket{i}.}
\end{equation}
Similarly, to multiply by a scalar, multiply all its components by the scalar:
\begin{equation}\label{component scalar multiplication}
    \boxed{a\ket{V} = a \sum_i v_i \ket{i} = \sum_i av_i \ket{i}.}
\end{equation}

\subsection{Inner Product Spaces}

\begin{shaded*}
The \df{inner product} between any two vectors, denoted $\braket{V}{W}$, is a number dependent on the two vectors which obeys the following:
\begin{enumerate}
    \item Skew-symmetry: $\braket{V}{W} = \braket{W}{V}^*$
    \item Positive semidefiniteness: $\braket{V}{V} \geq 0$ and is 0 \textit{iff} $\ket{V} = \ket{0}$
    \item Linearity in ket:
    \begin{align*}
        \bra{V} (a\ket{W} + b\ket{Z}) &\equiv \braket{V}{aW + bZ} \\
        &= a\braket{V}{W} + b\braket{V}{Z}.
    \end{align*}
\end{enumerate}
\end{shaded*}

A vector space with an inner product is called an \df{inner product space}.

From the first axiom, we have:
\begin{align}\label{antilinearity in bra}
    \braket{aW + bZ}{V} &= \braket{V}{aW + bZ}^* \nonumber \\
    &= (a\braket{V}{W} + b\braket{V}{Z})^* \nonumber \\
    &= a^*\braket{V}{W}^* + b^*\braket{V}{Z}^* \nonumber \\
    &= a^*\braket{W}{V} + b^*\braket{Z}{V}. \text{ That is,} \nonumber \\
    \Aboxed{\braket{aW + bZ}{V} &= a^*\braket{W}{V} + b^*\braket{Z}{V},}
\end{align}
which expresses \df{antilinearity} of the inner product with respect to the first factor in the inner product.

We say that two vectors are \df{orthogonal} if their inner product vanishes.

We will refer to $\sqrt{\braket{V}{V}} \equiv |V|$ as the \df{norm} of the vector. A \df{normalized vector} has unit norm.

A set of basis vectors all of unit norm, which are pairwise orthogonal will be called an \df{orthonormal basis}.

\begin{theorem}[Gram-Schmidt]\label{gram-schmidt}
Given a linearly independent basis we can form linear combinations of the basis vectors to obtain an orthonormal basis.
\end{theorem}

\begin{remark}
The Gram-Schmidt algorithm will be given in the next section.
\end{remark}

Now let us find an explicit formula for the inner product. Given $\ket{V} = \sum_i v_i \ket{i}$ and $\ket{W} = \sum_j w_j \ket{j}$, we obtain
\begin{equation*}
    \braket{V}{W} = \sum_i \sum_j v_i^* w_j \braket{i}{j}.
\end{equation*}
Assuming an orthonormal basis ($\braket{i}{j} = \kron$), we find
\begin{equation}\label{component inner product}
    \boxed{\braket{V}{W} = \sum_i v_i^* w_i.}
\end{equation}

\begin{remark}
Note that without the conjugation of the components of the first vector, $\braket{V}{V}$ would not necessarily be real or positive.
\end{remark}

Consider \eqref{component inner product}. Since $\ket{V}$ is uniquely specified by its components in a given basis, we may, in this basis, write it as a column vector:
\begin{equation}\label{ket as column}
    \ket{V} \rightarrow
    \begin{bmatrix}
    v_1 \\
    v_2 \\
    \vdots \\
    v_n
    \end{bmatrix}
    \text{ in this basis.}
\end{equation}
The inner product $\braket{V}{W}$ is given by the matrix product of the transpose conjugate of the column vector representing $\ket{V}$ with the column vector representing $\ket{W}$:
\begin{equation}\label{matrix inner product}
\boxed{
    \braket{V}{W} =
    \begin{bmatrix}
    v_1^* & v_2^* & \dots & v_n^*
    \end{bmatrix}
    \begin{bmatrix}
    w_1 \\
    w_2 \\
    \vdots \\
    w_n
    \end{bmatrix}.
}
\end{equation}

\begin{remark}
Given two linearly dependent vectors, you can use the inner product to find the multiple between them:
    $$\alpha\ket{V} + \beta\ket{W} = \ket{0}$$
    $$\implies \alpha\braket{V}{V} + \beta\braket{V}{W} = \braket{V}{0} = 0$$
    $$\implies \alpha = -\beta\frac{\braket{V}{W}}{\braket{V}{V}}.$$
You can then choose $\beta = -1$, and thus
\begin{equation}\label{two vector multiple}
    \boxed{\alpha = \frac{\braket{V}{W}}{\braket{V}{V}}.}
\end{equation}
\end{remark}

\subsection{Dual Spaces and the Dirac Notation}

We've been considering the inner product to be generated from two ket vectors, represented by column vectors. But there's no way to make a number out of two columns by direct matrix multiplication. We've been writing one of the kets as its conjugate transpose and then multiply the row and column.

More precisely, what's happening is that there are two vector spaces. According to \href{http://mathworld.wolfram.com/DualVectorSpace.html}{Wolfram MathWorld}, the dual vector space to a complex vector space $\vecsp$, denoted $\vecsp^*$, is the vector space of linear functionals $f : \vecsp \rightarrow \complex$. Bras are vectors in \emfu{this} dual vector space, \emfu{not} in the same vector space as the kets. We have these replacements:
\begin{align*}
    v &\rightarrow \ket{v}, \\
    \phi &\rightarrow \bra{u}, \\
    \phi_u(v) &\rightarrow \braket{u}{v}.
\end{align*}
It is the \emfu{functional} which is being represented by the row vector, because for each linear functional which takes a vector to a complex number there is a unique row vector which can be associated with it.

\subsubsection{Expansion of Vectors in an Orthonormal Basis}

Suppose we wish to expand a vector $\ket{V}$ in an orthonormal basis. We find
    $$\ket{V} = \sum_i v_i \ket{i}$$
    $$\implies \braket{j}{V} = \sum_i v_i \braket{j}{i} = v_j$$
    $$\implies \ket{V} = \sum_i \ket{i} \braket{i}{V},$$
since $\ket{i}$ and $\ket{j}$ are orthonormal.

That is,
\begin{equation}\label{component formula}
    \boxed{v_j = \braket{j}{V}}
\end{equation}

\subsubsection{Adjoint Operation}

If $\bra{V}$ is the bra corresponding to $\ket{V}$, what bra corresponds to $a\ket{V}$? In any basis, we find:
\begin{equation}\label{adjoint of multiplied ket}
    a\ket{V} \rightarrow
    \begin{bmatrix}
    av_1 \\
    av_2 \\
    \vdots \\
    av_n
    \end{bmatrix}
    \rightarrow
    \begin{bmatrix}
    a^*v_1^*, a^*v_2^*, \dots, a^*v_n^*
    \end{bmatrix}
    \rightarrow \bra{V}a^*.
\end{equation}

Thus,
\begin{align*}
    a\ket{V} &= b\ket{W} + c\ket{Z} + \dots \\
    \implies \bra{V}a^* &= \bra{W}b^* + \bra{Z}c^* + \dots.
\end{align*}
These two equations are said to be \df{adjoints of each other}. As a result, we can write
\begin{equation}\label{adjoint expansion}
    \boxed{\bra{V} = \sum_{i = 1} \bra{i}v_i^*.}
\end{equation}

\subsubsection{Gram-Schmidt Theorem}

\begin{shaded*}
The basic idea can be seen with the example of two linearly independent arrows:
\begin{enumerate}
    \item Scale the first by its own length, so it becomes a unit vector.
    \item Subtract from the second vector its projection along the first, leaving behind only the part perpendicular to the first.
    \item Scale the leftover piece by its own length. We now have an orthonormal basis.
\end{enumerate}
\end{shaded*}

\begin{proof}
\emfc{\textit{(Gram-Schmidt)}} Let $\ket{I}, \ket{II}, \dots$ be a linearly independent basis. The first vector of the orthonormal basis will be $$\ket{1} = \frac{\ket{I}}{|I|} \text{ where } |I| = \sqrt{\braket{I}{I}}.$$ Now consider $\ket{2'} = \ket{II} - \ket{1}\braket{1}{II}$. By construction it is orthogonal to $\ket{1}$: \\$\braket{1}{2'} = \braket{1}{II} - \braket{1}{1}\braket{1}{II} = 0.$ Then $$\ket{2} = \frac{\ket{2'}}{|2'|}.$$ Now consider $\ket{3'} = \ket{III} - \ket{1}\braket{1}{III} - \ket{2}\braket{2}{III}$, which is both orthogonal to $\ket{1}$ and $\ket{2}$. Then $$\ket{3} = \frac{\ket{3'}}{|3'|}.$$ Repeating this process, you have an orthonormal basis.
\end{proof}

\begin{exercise}
Form an orthonormal basis in two dimensions starting with $\vec{A} = 3\vec{i} + 4\vec{j}$ and $\vec{B} = 2\vec{i} - 6\vec{j}$. Can you generate another orthonormal basis starting with these two vectors? If so, produce another.
\end{exercise}

\begin{proof}
We first take
\begin{equation*}
   \ket{1} = \frac{\vec{A}}{\sqrt{3^2 + 4^2}} = 
   \begin{bmatrix}
   3/5 \\
   4/5
   \end{bmatrix}.
\end{equation*}
Then
\begin{align*}
   \ket{2'} &=
   \begin{bmatrix}
   2 \\
   -6
   \end{bmatrix}
   - 
   \begin{bmatrix}
   3/5 \\
   4/5
   \end{bmatrix}
   \begin{bmatrix}
   3/5 & 4/5
   \end{bmatrix}
   \begin{bmatrix}
   2 \\
   -6
   \end{bmatrix} \\
   &= 
   \begin{bmatrix}
   2 \\
   -6
   \end{bmatrix}
   - 
   \begin{bmatrix}
   3/5 \\
   4/5
   \end{bmatrix}
   (-18/5) \\
   &= 
   \begin{bmatrix}
   104/25 \\
   -78/25
   \end{bmatrix}.
\end{align*}
And so
\begin{align*}
    \ket{2} &= \frac{\ket{2'}}{\sqrt{(104/25)^2 + (-78/25)^2}} \\
    &= 
    \begin{bmatrix}
   104/25 \\
   -78/25
   \end{bmatrix}
   (5/26) \\
   &= 
   \begin{bmatrix}
   4/5 \\
   -3/5
   \end{bmatrix}.
\end{align*}
Thus $\ket{1}$ and $\ket{2}$ form an orthonormal basis.

Another orthonormal basis can be generated from these two vectors. This time we first take
\begin{equation*}
    \ket{1} = \frac{\vec{B}}{\sqrt{2^2 + (-6)^2}} = 
    \begin{bmatrix}
    1/\sqrt{10} \\
    -3/\sqrt{10}
    \end{bmatrix}.
\end{equation*}
Then 
\begin{align*}
    \ket{2'} &= 
    \begin{bmatrix}
    3 \\
    4
    \end{bmatrix}
    -
    \begin{bmatrix}
    1/\sqrt{10} \\
    -3/\sqrt{10}
    \end{bmatrix}
    \begin{bmatrix}
    1/\sqrt{10} & -3/\sqrt{10}
    \end{bmatrix}
    \begin{bmatrix}
    3 \\
    4
    \end{bmatrix} \\
    &= 
    \begin{bmatrix}
    3 \\
    4
    \end{bmatrix}
    -
    \begin{bmatrix}
    1/\sqrt{10} \\
    -3/\sqrt{10}
    \end{bmatrix}
    (-9/\sqrt{10}) \\
    &= 
    \begin{bmatrix}
    39/10 \\
    13/10
    \end{bmatrix}.
\end{align*}
And so
\begin{align*}
    \ket{2} &= \frac{\ket{2'}}{\sqrt{(39/10)^2 + (13/10)^2}} \\
    &= 
    \begin{bmatrix}
    39/10 \\
    13/10
    \end{bmatrix}
    (\sqrt{10}/13) \\
    &= 
    \begin{bmatrix}
    3/\sqrt{10} \\
    1/\sqrt{10}
    \end{bmatrix}.
\end{align*}
Thus $\ket{1}$ and $\ket{2}$ form an orthonormal basis.
\end{proof}

\begin{exercise}
Show how to go from the basis
\begin{equation*}
    \ket{I} = 
    \begin{bmatrix}
    3 \\
    0 \\
    0
    \end{bmatrix}
    \quad
    \ket{II} = 
    \begin{bmatrix}
    0 \\
    1 \\
    2
    \end{bmatrix}
    \quad
    \ket{III}
    \begin{bmatrix}
    0 \\
    2 \\
    5
    \end{bmatrix}
\end{equation*}
to the orthonormal basis
\begin{equation*}
    \ket{1} = 
    \begin{bmatrix}
    1 \\
    0 \\
    0
    \end{bmatrix}
    \quad
    \ket{2} = 
    \begin{bmatrix}
    0 \\
    1/\sqrt{5} \\
    2/\sqrt{5}
    \end{bmatrix}
    \quad
    \ket{3}
    \begin{bmatrix}
    0 \\
    -2/\sqrt{5} \\
    1/\sqrt{5}
    \end{bmatrix}.
\end{equation*}
\end{exercise}

\begin{proof}
We first take 
\begin{equation*}
    \ket{1} = \frac{\ket{I}}{\sqrt{3^2}} = 
    \begin{bmatrix}
    1 \\
    0 \\
    0
    \end{bmatrix}.
\end{equation*}
Then
\begin{align*}
    \ket{2'} &= 
    \begin{bmatrix}
    0 \\
    1 \\
    2
    \end{bmatrix}
    -
    \begin{bmatrix}
    1 \\
    0 \\
    0
    \end{bmatrix}
    \begin{bmatrix}
    1 & 0 & 0
    \end{bmatrix}
    \begin{bmatrix}
    0 \\
    1 \\
    2
    \end{bmatrix} \\
    &=
    \begin{bmatrix}
    0 \\
    1 \\
    2
    \end{bmatrix}
\end{align*}
And so
\begin{align*}
    \ket{2} = \frac{\ket{2'}}{\sqrt{1^2 + 2^2}} = 
    \begin{bmatrix}
    0 \\
    1/\sqrt{5} \\
    2/\sqrt{5}
    \end{bmatrix}.
\end{align*}
Then
\begin{align*}
    \ket{3'} = 
    \begin{bmatrix}
    0 \\
    2 \\
    5
    \end{bmatrix}
    &-
    \begin{bmatrix}
    1 \\
    0 \\
    0
    \end{bmatrix}
    \begin{bmatrix}
    1 & 0 & 0
    \end{bmatrix}
    \begin{bmatrix}
    0 \\
    2 \\
    5
    \end{bmatrix} \\
    &-
    \begin{bmatrix}
    0 \\
    1/\sqrt{5} \\
    2/\sqrt{5}
    \end{bmatrix}
    \begin{bmatrix}
    0 & 1/\sqrt{5} & 2/\sqrt{5}
    \end{bmatrix}
    \begin{bmatrix}
    0 \\
    2 \\
    5
    \end{bmatrix} \\
    =
    \begin{bmatrix}
    0 \\
    2 \\
    5
    \end{bmatrix}
    &-
    \begin{bmatrix}
    0 \\
    1/\sqrt{5} \\
    2/\sqrt{5}
    \end{bmatrix}
    (12/\sqrt{5}) \\
    = 
    \begin{bmatrix}
    0 \\
    2 \\
    5
    \end{bmatrix}
    &- 
    \begin{bmatrix}
    0 \\
    12/5 \\
    24/5
    \end{bmatrix} 
    =
    \begin{bmatrix}
    0 \\
    -2/5 \\
    1/5
    \end{bmatrix}
\end{align*}
And so
\begin{align*}
    \ket{3} &= \frac{\ket{3'}}{\sqrt{(-2/5)^2 + (1/5)^2}} \\
    &= 
    \begin{bmatrix}
    0 \\
    -2/5 \\
    1/5
    \end{bmatrix}
    (5/\sqrt{5}) \\
    &= 
    \begin{bmatrix}
    0 \\
    -2/\sqrt{5} \\
    1/\sqrt{5}
    \end{bmatrix}.
\end{align*}
\end{proof}

\begin{theorem}\label{n_perp dim}
The dimensionality of a space equals $n_\perp$, the maximum number of mutually orthogonal vectors in it.
\end{theorem}

\begin{proof}
Any mutually orthogonal set is linearly independent:
    $$\ket{0} = a\ket{1} + b\ket{2} + \dots$$
    $$\implies \braket{1}{0} = a\braket{1}{1} + b\braket{1}{2} + \dots$$
    $$0 = a\braket{1}{1}$$
    $$\implies a = 0.$$
Since we could have taken the inner product with any of the orthogonal vectors, the only solution to \eqref{linear independence relation} is the trivial one.

Note that $n_\perp \nless n$, the dimensionality of the space, by the explicit construction in the Gram-Schmidt procedure. Moreover, the linear independence of the orthogonal vectors means $n_\perp \ngtr n$. Thus $n_\perp = n$.
\end{proof}

\subsubsection{Schwarz and Triangle Inequalities}

\begin{theorem}[The Schwarz Inequality]\label{schwarz inequality}
The Schwarz Inequality states that $|\braket{V}{W}| \leq |V||W|$.
\end{theorem}

\begin{proof}
Consider
\begin{equation*}
    \ket{Z} = \ket{V} - \frac{\braket{W}{V}}{|W|^2}\ket{W}.
\end{equation*}
Then, since $\braket{Z}{Z} \geq 0$,
\begin{align*}
    \braket{Z}{Z} &= \bra{Z} \bigg{(}\ket{V} - \frac{\braket{W}{V}}{|W|^2}\ket{W}\bigg{)} \\
    &= \braket{Z}{V} - \frac{\braket{W}{V}}{|W|^2} \braket{Z}{W} \\
    &= \bigg{(}\bra{V} - \bra{W}\frac{\braket{W}{V}^*}{|W|^2}\bigg{)}\ket{V}  \\
    &\qquad - \frac{\braket{W}{V}}{|W|^2} \bigg{(}\bra{V} - \bra{W}\frac{\braket{W}{V}^*}{|W|^2}\bigg{)} \ket{W} \\
    &= \braket{V}{V} - \frac{\braket{W}{V}}{|W|^2} \braket{W}{V} \\
    &\qquad - \frac{\braket{W}{V}}{|W|^2}\bigg{(}\braket{V}{W} - \frac{\braket{W}{V}}{|W|^2}\braket{W}{W}\bigg{)} \\
    &= \braket{V}{V} - \frac{\braket{W}{V} \braket{V}{W}}{|W|^2} \\
    &\geq 0 \\
    \implies &\braket{V}{V} \geq \frac{\braket{W}{V} \braket{V}{W}}{|W|^2} \\
    \implies &|\braket{V}{W}| \leq |V||W|.
\end{align*}
\end{proof}

\begin{exercise}
When will this equality be satisfied? Does this agree with your experience of arrows?
\end{exercise}

\begin{proof}
The equality holds if and only if $\ket{V}$ and $\ket{W}$ are linearly dependent.

First suppose that $\ket{V} = a\ket{W}$. Then
\begin{align*}
    |\braket{V}{W}| &= |a^*\braket{W}{W}| \\
    &= |a||W||W| \\
    &= |V||W|.
\end{align*}

Now suppose that $|\braket{V}{W}| = |V||W|$. Then $\braket{V}{W} \neq 0$, and so the vectors are linearly dependent by \eqref{two vector multiple}.

This agrees with my experience of arrows because that's the only time $\cos{(\theta)} = 1$.
\end{proof}

\begin{theorem}[The Triangle Inequality]\label{triangle inequality}
The Triangle Inequality states that $|V + W| \leq |V| + |W|$.
\end{theorem}

\begin{exercise}
Prove the triangle inequality starting with $|V + W|^2$. You must use $\real \braket{V}{W} \leq |\braket{V}{W}|$ and the Schwarz inequality. Show that the final inequality becomes an equality only if $\ket{V} = a\ket{W}$ where $a$ is a real positive scalar.
\end{exercise}

\begin{proof} We have
\begin{align}\label{triangle equality}
    |V + W|^2 &= \braket{V + W}{V + W} \nonumber \\
    &= \braket{V}{V} + \braket{V}{W} + \braket{W}{V} + \braket{W}{W} \nonumber \\
    &= \braket{V}{V} + \braket{V}{W} + \braket{V}{W}^* + \braket{W}{W} \nonumber \\
    &= \braket{V}{V} + 2\real\braket{V}{W} + \braket{W}{W} \nonumber \\
    &\leq \braket{V}{V} + 2|\braket{V}{W}| + \braket{W}{W} \\
    &\leq |V|^2 + 2|V||W| + |W|^2 \nonumber \\
    &= |V|^2 + |W|^2. \nonumber
\end{align}

Suppose $|V + W|^2 = |V|^2 + |W|^2$. Then \eqref{triangle equality} becomes
\begin{equation*}
    |V|^2 + 2\real \braket{V}{W} + |W|^2 = |V|^2 + 2|\braket{V}{W}| + |W|^2.
\end{equation*}
That is,
\begin{equation*}
    \real \braket{V}{W} = |\braket{V}{W}|,
\end{equation*}
which implies that $\braket{V}{W}$ is a real positive scalar. By \eqref{two vector multiple} $a$ is a real positive scalar.
\end{proof}

\subsection{Subspaces}

Given a vector space $\vecsp$, a subset of its elements that form a vector space among themselves (under the same vector addition and scalar multiplication) is called a \df{subspace}. We will denote a particular subspace $i$ of dimensionality $n_i$ by $\vecsp_i^{n_i}$.

\begin{example}
The space $\vecsp^3(\reals)$ has the following subspaces:
\begin{enumerate}
    \item all vectors along the $x$ axis, $\vecsp_x^{1}$
    \item all vectors along the $y$ axis, $\vecsp_y^{1}$
    \item all vectors in the $x-y$ plane, $\vecsp_{xy}^{2}$.
\end{enumerate}
\end{example}

\begin{shaded*}
Given two subspaces $\vecsp_i^{n_i}$ and $\vecsp_j^{m_j}$, we define their sum $\vecsp_i^{n_i} \oplus \vecsp_j^{m_j} = \vecsp_k^{m_k}$ as the set containing
\begin{enumerate}
    \item all elements of $\vecsp_i^{n_i}$
    \item all elements of $\vecsp_j^{m_j}$
    \item all possible linear combinations of the above.
\end{enumerate}
\end{shaded*}

\begin{example}
$\vecsp_x^{1} \oplus \vecsp_y^{1} = \vecsp_{xy}^{2}$.
\end{example}

\begin{exercise}
In a space $\vecsp^n$, prove that the set of all vectors $\{\ket{V_\perp^1}, \ket{V_\perp^2}, \dots\}$, orthogonal to any \\$\ket{V} \neq \ket{0}$, form a subspace $\vecsp^{n - 1}$.
\end{exercise}

\begin{proof}
Let $\ket{V} \neq \ket{0}$ be arbitrary. First we check closure:
\begin{align*}
    \braket{V_\perp^i + V_\perp^j}{V} = \braket{V_\perp^i}{V} + \braket{V_\perp^j}{V} = 0 \quad &\checkmark \\
    \braket{aV_\perp^i}{V} = a^*\braket{V_\perp^i}{V} = 0. \quad &\checkmark
\end{align*}

Next we check that the null vector is in the set:
\begin{equation*}
    \braket{0}{V} = 0. \quad \checkmark
\end{equation*}

Next we check that $\ket{-V_\perp^i}$ is in the set:
\begin{equation*}
    \braket{-V_\perp^i}{V} = -\braket{V_\perp^i}{V} = 0. \quad \checkmark
\end{equation*}

All other axioms of a vector space carry over, and so the set is a subspace.

To see that the dimensionality of the space is $n - 1$, consider an orthogonal basis for $\vecsp^n$ for which $\ket{V}$ is an element. Then it is clear that the other $n - 1$ elements of the basis form an orthogonal basis for $\{\ket{V_\perp^1}, \ket{V_\perp^2}, \dots\}$.
\end{proof}

\begin{exercise}
Suppose $\vecsp_1^{n_1}$ and $\vecsp_2^{n_2}$ are two subspaces such that any element of $\vecsp_1$ is orthogonal to any element of $\vecsp_2$. Show that the dimensionality of $\vecsp_1 \oplus \vecsp_2$ is $n_1 + n_2$. (Hint: Theorem \eqref{n_perp dim}.)
\end{exercise}

\begin{proof}
Let $\ket{1}, \ket{2}, \dots, \ket{n_1}$ be an orthogonal basis for $\vecsp_1$ and $\ket{1'}, \ket{2'}, \dots, \ket{n_2'}$ be an orthogonal basis for $\vecsp_2$. First note that since any element of $\vecsp_1$ is orthogonal to any element of $\vecsp_2$, each element of the basis for $\vecsp_1$ is orthogonal to each element of the basis for $\vecsp_2$. Thus we can form the orthogonal basis $\ket{1}, \ket{2}, \dots, \ket{n_1}, \ket{1'}, \ket{2'}, \dots, \ket{n_2'}$ for $\vecsp_1 \oplus \vecsp_2$.
\end{proof}

\subsection{Linear Operators}

An \df{operator} $\Omega$ is an instruction for transforming any given vector $\ket{V}$ into another, $\ket{V'}$. The action of the operator is represented as follows: $\Omega\ket{V} = \ket{V'}$. The can also act on bras: $\bra{V'}\Omega = \bra{V''}$.
\begin{shaded*}
We will only be concerned with \df{linear operators}, which obey the following:
\begin{enumerate}
    \item $\Omega \alpha \ket{V_i} = \alpha\Omega\ket{V_i}$
    \item $\Omega\{\alpha\ket{V_i} + \beta\ket{V_j}\} = \alpha\Omega\ket{V_i} + \beta\Omega\ket{V_j}$
    \item $\bra{V_i}\alpha\Omega = \bra{V_i}\Omega\alpha$
    \item $\{\bra{V_i}\alpha + \bra{V_j}\beta\}\Omega = \alpha\bra{V_i}\Omega + \beta\bra{V_j}\Omega$.
\end{enumerate}
\end{shaded*}

\begin{example}
The identity operator $I$, whose action is $I\ket{V} = \ket{V}$ for all kets $\ket{V}$ and $\bra{V}I = \bra{V}$ for all bras $\bra{V}$.
\end{example}

\begin{example}
The operator on $\vecsp^3(\reals)$ given by the action: $$R(\bm{\theta}) \rightarrow \text{Rotate vector by } |\bm{\theta}| \text{ about unit vector } \tfrac{\bm{\theta}}{|\bm{\theta}|}.$$ For instance:
\begin{align*}
    R(\tfrac{1}{2}\pi \bm{i})\ket{1} &= \ket{1} \\
    R(\tfrac{1}{2}\pi \bm{i})\ket{2} &= \ket{3} \\
    R(\tfrac{1}{2}\pi \bm{i})\ket{3} &= -\ket{2}.
\end{align*}
\end{example}

Linear operators are nice because once their action on the basis vectors is known, their action on any vector in the space is determined. If $\Omega\ket{i} = \ket{i'}$ for a basis, then for any $\ket{V} = \sum v_i \ket{i}$
\begin{equation}\label{linear operator action on arbitrary vector}
    \boxed{\Omega\ket{V} = \sum_i \Omega v_i\ket{i} = \sum_i v_i \Omega\ket{i} = \sum_i v_i\ket{i'}.}
\end{equation}

The \df{product of two operators} stands for the instruction that the instructions corresponding to the two operators be carried out in sequence:
\begin{equation}\label{operator product}
    \boxed{\Lambda\Omega\ket{V} = \Lambda(\Omega\ket{V}) = \Lambda\ket{\Omega V}.}
\end{equation}

In general,
\begin{equation}\label{commutator}
    \boxed{\Omega\Lambda - \Lambda\Omega \equiv [\Omega, \Lambda] \neq 0.}
\end{equation}
The object $[\Omega, \Lambda]$ is called the \df{commutator} of $\Omega$ and $\Lambda$.

Two useful identities involving commutators are:
\begin{align}\label{commutator identities}
    [\Omega, \Lambda \theta] &= \Lambda[\Omega, \theta] + [\Omega, \Lambda]\theta \\
    [\Lambda \Omega, \theta] &= \Lambda[\Omega, \theta] + [\Lambda, \theta]\Omega. \nonumber
\end{align}
Note the similarity with the chain rule for derivatives.

The \df{inverse of an operator} $\Omega$, $\Omega^{-1}$, satisfies
\begin{equation}\label{operator inverse}
    \boxed{\Omega\Omega^{-1} = \Omega^{-1}\Omega = I.}
\end{equation}
Not every operator has an inverse (see \eqref{matrix inversion}).

We have
\begin{equation}\label{operator product inverse}
    \boxed{(\Omega\Lambda)^{-1} = \Lambda^{-1}\Omega^{-1}}
\end{equation}
for only then do we have
\begin{align*}
    (\Omega\Lambda)(\Omega\Lambda)^{-1} &= (\Omega\Lambda)(\Lambda^{-1}\Omega^{-1}) \\
    &= \Omega\Lambda\Lambda^{-1}\Omega^{-1} \\
    &= \Omega\Omega^{-1} = I.
\end{align*}

\subsection{Matrix Elements of Linear Operators}

\subsection{Active and Passive Transformations}

\subsection{The Eigenvalue Problem}

\subsection{Functions of Operators and Related Concepts}

\subsection{Generalization to Infinite Dimensions}


\section{Review of Classical Mechanics}
\subsection{The Principle of Least Action and Lagrangian Mechanics}

\subsection{The Electromagnetic Lagrangian}

\subsection{The Two-Body Problem}

\subsection{How Smart Is a Particle?}

\subsection{The Hamiltonian Formalism}

\subsection{The Electromagnetic Force in the Hamiltonian Scheme}

\subsection{Cyclic Coordinates, Poisson Brackets, and Canonical Transformations}

\subsection{Symmetries and Their Consequences}


\section{All Is Not Well with Classical Mechanics}
\subsection{Particles and Waves in Classical Physics}

\subsection{An Experiment with Waves and Particles (Classical)}

\subsection{The Double-Slit Experiment with Light}

\subsection{Matter Waves (de Broglie Waves)}

\subsection{Conclusions}


\section{The Postulates--A General Discussion}
\subsection{The Postulates}

\subsection{Discussion of Postulates I-III}

\subsection[The Schr\"odinger Equation]{The Schr\"odinger Equation (Dotting Your $i$'s and Crossing your $\hbar$'s)}


\section{Simple Problems in One Dimension}
\subsection{The Free Particle}

\subsection{The Particle in a Box}

\subsection{The Continuity Equation for Probability}

\subsection{The Single-Step Potential: a Problem in Scattering}

\subsection{The Double-Slit Experiment}

\subsection{Some Theorems}


\section{The Classical Limit}
\input{chapters/ch06.tex}

\section{The Harmonic Oscillator}
\subsection{Why Study the Harmonic Oscillator}

\subsection{Review of the Classical Oscillator}

\subsection{Quantization of the Oscillator (Coordinate Basis)}

\subsection{The Oscillator in the Energy Basis}

\subsection[Passage from the Energy Basis to the Coordinate Basis]{Passage from the Energy Basis to the $X$ Basis}


\section{The Path Integral Formulation of Quantum Theory}
\subsection{The Path Integral Recipe}

\subsection{Analysis of the Recipe}

\subsection[An Approximation to the Propagator for the Free Particle]{An Approximation to $U(t)$ for the Free Particle}

\subsection{Path Integral Evaluation of the Free-Particle Propagator}

\subsection{Equivalence to the Sch\"odinger Equation}

\subsection[A Class of Potentials]{Potentials of the Form $V = a + bx + cx^2 + d\dot{x} + ex\dot{x}$}


\section{The Heisenberg Uncertainty Relation}
\subsection{Introduction}

\subsection{Derivation of the Uncertainty Relations}

\subsection{The Minimum Uncertainty Packet}

\subsection{Applications of the Uncertainty Principle}

\subsection{The Energy-Time Uncertainty Relation}


\section{System with $N$ Degrees of Freedom}
\subsection{$N$ Particles in One Dimension}

\subsection{More Particles in More Dimensions}

\subsection{Identical Particles}


\section{Symmetries and their Consequences}
\subsection{Overview}

\subsection{Translational Invariance in Quantum Theory}

\subsection{Time Translational Invariance}

\subsection{Parity Invariance}

\subsection{Time-Reversal Symmetry}


\section{Rotational Invariance and Angular Momentum}
\subsection{Translations in Two Dimensions}

\subsection{Rotations in Two Dimensions}

\subsection[The Eigenvalue Problem of Lz]{The Eigenvalue Problem of $L_z$}

\subsection{Angular Momentum in Three Dimensions}

\subsection[The Eigenvalue Problem of L2 and Lz]{The Eigenvalue Problem of $L^2$ and $L_z$}

\subsection{Solution of Rotationally Invariant Problems}


\section{The Hydrogen Atom}
\subsection{The Eigenvalue Problem}

\subsection{The Degeneracy of the Hydrogen Spectrum}

\subsection{Numerical Estimates and Comparison with Experiment}

\subsection{Multielectron Atoms and the Periodic Table}


\section{Spin}
\subsection{Introduction}

\subsection{What is the Nature of Spin?}

\subsection{Kinematics of Spin}

\subsection{Spin Dynamics}

\subsection{Return of Orbital Degrees of Freedom}


\section{Addition of Angular Momentum}
\subsection{A Simple Example}

\subsection{The General Problem}

\subsection{Irreducible Tensor Operators}

\subsection{Explanation of Some ``Accidental" Degeneracies}


\section{Variational and WKB Methods}
\subsection{The Variational Method}

\subsection{The Wentzel-Kramers-Brillouin Method}


\section{Time-Independent Perturbation Theory}
\subsection{The Formalism}

\subsection{Some Examples}

\subsection{Degenerate Perturbation Theory}


\section{Time-Dependent Perturbation Theory}
\subsection{The Problem}

\subsection{First-Order Perturbation Theory}

\subsection{Higher Orders in Perturbation Theory}

\subsection{A General Discussion of Electromagnetic Interactions}

\subsection{Interaction of Atoms with Electromagnetic Radiation}


\section{Scattering Theory}
\subsection{Introduction}

\subsection{Recapitulation of One-Dimensional Scattering and Overview}

\subsection{The Born Approximation (Time-Dependent Description)}

\subsection{Born Again (The Time-Independent Approximation)}

\subsection{The Partial Wave Expansion}

\subsection{Two-Particle Scattering}


\section{The Dirac Equation}
\subsection{The Free-Particle Dirac Equation}

\subsection{Electromagnetic Interaction of the Dirac Particle}

\subsection{More on Relativistic Quantum Mechanics}


\section{Path Integrals--II}
\subsection{Derivation of the Path Integral}

\subsection{Imaginary Time Formalism}

\subsection{Spin and Fermion Path Integrals}

\subsection{Summary}


\appendix
\section{Appendix}

\subsection{Matrix Inversion}\label{matrix inversion}

Consider the matrix
\begin{equation}
    M = 
    \begin{bmatrix}
    a_1 & a_2 & a_3 \\
    b_1 & b_2 & b_3 \\
    c_1 & c_2 & c_3
    \end{bmatrix}\,.
\end{equation}
We consider the rows as vectors: $\bm{A} = a_1\bm{i} + a_2\bm{j} + a_3\bm{k}$, and similarly for the other two rows. Recall that
\begin{equation}\label{cross product}
    \tcboxmath{(\bm{V \times W})^i = \varepsilon_{ijk}v^jw^k\,.}
\end{equation}

Now consider
\begin{align}
    \bm{A}_R &= \bm{B} \times \bm{C} \\
    \bm{B}_R &= \bm{C} \times \bm{A} \\
    \bm{C}_R &= \bm{A} \times \bm{B}\,,
\end{align}
which are said to be \df{reciprocal} to $\bm{A}$, $\bm{B}$, and $\bm{C}$. In general, 
\begin{equation}
\bm{A} \cdot \bm{A}_R \neq 0\,, \quad \bm{A} \cdot \bm{B}_R = \bm{A} \cdot \bm{C}_R = 0\,,
\end{equation}
and cyclic permutations. For example:
\begin{multline}
    \bm{A} \cdot \bm{A}_R = a_1(b_2c_3 - b_3c_2) + a_2(b_3c_1 - b_1c_3) \\
    + a_3(b_1c_2 - b_2c_1) = \det M\,.
\end{multline}
while 
\begin{align}
    \bm{A} \cdot \bm{B}_R = a_1(c_2a_3 - c_3a_2) &+ a_2(c_3a_1 - c_1a_3) \\
    &+ a_3(c_1a_2 - c_2a_1) = 0\,.
\end{align}

We now construct a matrix called the \df{cofactor transpose} of $M$ whose columns are the reciprocal vectors:
\begin{equation*}
    \bar{M} = 
    \begin{bmatrix}
    (a_R)_1 & (b_R)_1 & (c_R)_1 \\
    (a_R)_2 & (b_R)_2 & (c_R)_2 \\
    (a_R)_3 & (b_R)_3 & (c_R)_3
    \end{bmatrix}
\end{equation*}
and note that 
\begin{align}
    M \bar{M} &= 
    \begin{bmatrix}
    \bm{A} \cdot \bm{A}_R & \bm{A} \cdot \bm{B}_R & \bm{A} \cdot \bm{C}_R \\
    \bm{B} \cdot \bm{A}_R & \bm{B} \cdot \bm{B}_R & \bm{B} \cdot \bm{C}_R \\
    \bm{C} \cdot \bm{A}_R & \bm{C} \cdot \bm{B}_R & \bm{C} \cdot \bm{C}_R
    \end{bmatrix} \\
    &=
    \begin{bmatrix}
    \bm{A} \cdot \bm{A}_R & 0 & 0 \\
    0 & \bm{B} \cdot \bm{B}_R & 0 \\
    0 & 0 & \bm{C} \cdot \bm{C}_R
    \end{bmatrix}\,. \nonumber
\end{align}

Thus, we've found
\begin{equation}\label{matrix inverse}
    \tcboxmath{M^{-1} = \frac{\bar{M}}{\det M}\,.}
\end{equation}

Note that $\det M$ vanishes if the rows of the matrix are not linearly independent: if, say, $\bm{C} = \alpha\bm{A} + \beta\bm{B}$, then
\begin{align}
    \bm{A} \cdot \bm{A}_R &= \bm{A} \cdot (\bm{B} \times \bm{C}) \\
    &= \bm{A} \cdot (\bm{B} \times \alpha \bm{A}) + \bm{A} \cdot (\bm{B} \times \beta\bm{B}) \\
    &= \bm{B} \cdot (\alpha\bm{A} \times \bm{A}) \\
    &= 0\,.
\end{align}

We can generalize this to invert $n \times n$ matrices. Define
\begin{equation}\label{cross product generalization}
    \tcboxmath{(\bm{A}_1 \times \bm{A}_2 \times \cdots \times \bm{A}_{n - 1})^i = \varepsilon_{i_1 \dots i_{n-1}} a_1^{i_1} \dots a_{n - 1}^{i_{n - 1}}\,.}
\end{equation}

Then if we have a matrix $M$ whose $n$ rows may be identified with $n$ vectors $\bm{A}_1, \bm{A}_2, \dots, \bm{A}_n$, then $\bar{M}$ has as its columns $\bm{A}_{1R}, \dots, \bm{A}_{nR}$.

\begin{exercise}
Using the method described above, show that
\begin{equation*}
    \begin{bmatrix}
    2 & 1 & 3 \\
    0 & 1 & 2 \\
    -1 & 1 & 1
    \end{bmatrix}^{-1}
    =
    \begin{bmatrix}
    1 & -2 & 1 \\
    2 & -5 & 4 \\
    -1 & 3 & -2
    \end{bmatrix}
\end{equation*}
and
\begin{equation*}
    \begin{bmatrix}
    2 & 1 & 3 \\
    4 & 1 & 2 \\
    0 & -1 & 2
    \end{bmatrix}^{-1}
    = \frac{1}{12}
    \begin{bmatrix}
    -4 & 5 & 1 \\
    8 & -4 & -8 \\
    4 & -2 & 2
    \end{bmatrix}\,.
\end{equation*}
\end{exercise}

\begin{proof}
For the first matrix, we have $$\bm{A} = (2, 1, 3) \quad \bm{B} = (0, 1, 2) \quad \text{and} \quad \bm{C} = (-1, 1, 1),$$ and so
\begin{align}
    \bm{A}_R &= (0, 1, 2) \times (-1, 1, 1) = (-1, -2, 1)\,, \\
    \bm{B}_R &= (-1, 1, 1) \times (2, 1, 3) = (2, 5, -3)\,, \\
    \bm{C}_R &= (2, 1, 3) \times (0, 1, 2) = (-1, -4, 2)\,.
\end{align}
Therefore,
\[ \det M = \bm{A} \cdot \bm{A}_R = -2 - 2 + 3 = -1\,. \]
Then \eqref{matrix inverse} gives the required matrix.

For the second matrix, we have
\[ \bm{A} = (2, 1, 3) \quad \bm{B} = (4, 1, 2) \quad \bm{C} = (0, -1, 2)\,, \]
and so
\begin{align}
    \bm{A}_R &= (4, 1, 2) \times (0, -1, 2) = (4, -8, -4)\,, \\
    \bm{B}_R &= (0, -1, 2) \times (2, 1, 3) = (-5, 4, 2)\,, \\
    \bm{C}_R &= (2, 1, 3) \times (4, 1, 2) = (-1, 8, -2)\,.
\end{align}
Therefore,
\[ \det M = \bm{A} \cdot \bm{A}_R = 8 - 8 - 12 = -12\,. \]
Then \eqref{matrix inversion} gives the required matrix.
\end{proof}

\begin{theorem}\label{th a1}
If $\Omega \ket{V} = \ket{0}$ implies $\ket{V} = \ket{0}$ then $\Omega^{-1}$ exists.
\end{theorem}

\begin{lemma}\label{linear operator on linearly independent basis}
If $\ket{V_1}, \dots, \ket{V_n}$ forms a linearly independent basis, then so does $\Omega\ket{V_1}, \dots, \Omega\ket{V_n}$.
\end{lemma}

\begin{proof}
Consider
\[ \ket{0} = \sum_i a_i\Omega\ket{V_i} = \Omega \bigg{(}\sum_i a_i\ket{V_i}\bigg{)}\,. \]
Then for all $i$, $a_i = 0$ since the $\ket{V_i}$ are linearly independent.
\end{proof}

\begin{proof}[Proof (Theorem \eqref{th a1})]

\end{proof}

\subsection{Gaussian Integrals}

\subsection{Complex Numbers}

\subsection[The Principle Value Integral]{The $i\epsilon$ Prescription}


\end{document}
